% Created: 2020-01-10, John Miller

%==========================================================
%=========== Document Setup  ==============================

% Formatting defined by class file
\documentclass[11pt]{article}

% ---- Document formatting ----
\usepackage[margin=1in]{geometry}	% Narrower margins
\usepackage{booktabs}				% Nice formatting of tables
\usepackage{graphicx}				% Ability to include graphics

%\setlength\parindent{0pt}	% Do not indent first line of paragraphs 
\usepackage[parfill]{parskip}		% Line space b/w paragraphs
%	parfill option prevents last line of pgrph from being fully justified

% Parskip package adds too much space around titles, fix with this
\RequirePackage{titlesec}
\titlespacing\section{0pt}{8pt plus 4pt minus 2pt}{3pt plus 2pt minus 2pt}
\titlespacing\subsection{0pt}{4pt plus 4pt minus 2pt}{-2pt plus 2pt minus 2pt}
\titlespacing\subsubsection{0pt}{2pt plus 4pt minus 2pt}{-6pt plus 2pt minus 2pt}

% ---- Hyperlinks ----
\usepackage[colorlinks=true,urlcolor=blue]{hyperref}	% For URL's. Automatically links internal references.

% ---- Code listings ----
\usepackage{listings} 					% Nice code layout and inclusion
\usepackage[usenames,dvipsnames]{xcolor}	% Colors (needs to be defined before using colors)

% Define custom colors for listings
\definecolor{listinggray}{gray}{0.98}		% Listings background color
\definecolor{rulegray}{gray}{0.7}			% Listings rule/frame color

% Style for Verilog
\lstdefinestyle{Verilog}{
	language=Verilog,					% Verilog
	backgroundcolor=\color{listinggray},	% light gray background
	rulecolor=\color{blue}, 			% blue frame lines
	frame=tb,							% lines above & below
	linewidth=\columnwidth, 			% set line width
	basicstyle=\small\ttfamily,	% basic font style that is used for the code	
	breaklines=true, 					% allow breaking across columns/pages
	tabsize=3,							% set tab size
	commentstyle=\color{gray},	% comments in italic 
	stringstyle=\upshape,				% strings are printed in normal font
	showspaces=false,					% don't underscore spaces
}

% How to use: \Verilog[listing_options]{file}
\newcommand{\Verilog}[2][]{%
	\lstinputlisting[style=Verilog,#1]{#2}
}




%======================================================
%=========== Body  ====================================
\begin{document}

\title{ELC 2137 Lab 6: MUX and 7-segment Decoder}
\author{Trevor Jackson, Carlos Hernandez, and Makenna Meyers}

\maketitle


\section*{Summary}

SUMMARY

\section*{Q\&A}

\begin{enumerate}
	\item How many wires are connected to the 7-segment display?
	
	ANSWER
	
	\item If the segments were not all connected together, how many wires would there have to be?

    ANSWER
    
    \item Why do we prefer the current method vs. separating all of the segments?
    
    ANSWER
    
    \item List the errors found during simulation. What does this tell you about why we run simulations?
    
    ANSWER
	
\end{enumerate}

\section*{Results}

\begin{figure}\centering
	\includegraphics[width=0.8\textwidth,trim=0cm 11cm 0cm 9.5cm,clip]{picture1}
	\caption{First Digit Value}
	\label{fig:picture1}	
\end{figure}

\begin{figure}\centering
	\includegraphics[width=0.8\textwidth,trim=0cm 7.5cm 0cm 7.5cm,clip]{picture2}
	\caption{Second Digit Value}
	\label{fig:picture2}	
\end{figure}

\clearpage

\section*{Code}

\Verilog[caption=Half Adder Code,label=code:half adder]{halfadder.sv}

\Verilog[caption=Half Adder Test Bench Code,label=code:half adder test bench]{halfadder_test.sv}

\Verilog[caption=Full Adder Code,label=code:full adder]{fulladder.sv}

\Verilog[caption=Full Adder Test Bench Code,label=code:full adder test bench]{fulladder_test.sv}

\Verilog[caption=Adder/Subtractor Code,label=code:adder/subtractor]{addersub2.sv}

\Verilog[caption=Adder/Subtractor Test Bench Code,label=code:adder/subtractor test bench]{addersub_test2.sv}

\end{document}
